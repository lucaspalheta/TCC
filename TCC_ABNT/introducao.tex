\chapter{Introdução}\label{CAP:introducao}
%\thispagestyle{empty}

Este documento consiste de um modelo basico para a producao de documentos academicos, seguindo as normas ABNT. 

Nao e abordado o estudo do LaTex neste template. Sugerimos a leitura do texto em \citeonline{Oetiker:1995}. O uso do LaTex e aconselhavel devido a sua qualidade grafica, facil referenciacao, criacao de listas, indices, referencias bibliograficas e escrita matematica profissional. Porem, nao e obrigatorio o uso deste template, apenas as orientacoes de formatação segundo as normas ABNT devem ser obrigatoriamente seguidas.

Uma observação em particular é a de que, no pacote ABNTex, as referências diretas devem utilizar o comando ``citeonline''. Referências indiretas utilizam o comando ``cite''.

Exemplo de citacao direta: Uma otima fonte de estudo para compreender o LaTex e apresentada por \citeonline{Oetiker:1995}. 

Exemplo de citação indireta: Existem boas fontes de pesquisa para entendimento do LaTex \cite{Oetiker:1995}, estas incluem documentação online disponível na web.

\section{Motivação e objetivos}


 
\section{Contribuicoes}




\section{Producao cientifica}


\section{Organizacao da tese}

\noindent \textbf{Capitulo \ref{CAP2}}: descricao...

\noindent \textbf{Capitulo \ref{CAP3}}: descricaoo...

\noindent \textbf{Capitulo \ref{CAP4}}: descricao...

\noindent \textbf{Capitulo \ref{CAP5}}: descricao...