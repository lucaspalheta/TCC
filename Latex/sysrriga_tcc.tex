\documentclass[12pt,a4paper]{book}
\usepackage[utf8]{inputenc}
\usepackage[portuguese]{babel}
\usepackage[T1]{fontenc}
\usepackage{amsmath}
\usepackage{amsfonts}
\usepackage{amssymb}
\makeindex
\usepackage{makeidx}
\usepackage{graphicx}
\usepackage[left=3cm,right=2cm,top=3cm,bottom=2cm]{geometry}
\author{Lucas Palheta Sampaio}
\title{Desenvolvimento de Ferramenta em Ambiente Web para Projetismo de Sistemas de Irrigação}
\begin{document}
\maketitle 
\tableofcontents

\chapter{Introdução}



\chapter{Irrigação}

A maior parte da água doce utilizada pelo homem é destinada para a agricultura. Por isso, sistemas de irrigação que permitam o uso racional desse recurso são essenciais para o desenvolvimento sustentável \cite{catalogotigre}.

	Segundo \cite{paz2002otimizaccao} a irrigação com maior uniformidade proporciona melhores resultados econômicos em qualquer combinação de preço do produto e custo da água.

A Irrigação localizada está entre as contribuições mais efetivas no cenário da agricultura, principalmente no âmbito da fruticultura irrigada no Brasil, seu reflexo pode ser observado desde o século passado, onde \cite{nascimento1999caracterizaccao} afirma que "a competição futura por água e energia elétrica, principalmente no vale do São Francisco, tenderá a priorizar o
emprego de sistemas de irrigação mais eficientes".



\section{Sistemas}
\subsection{Componentes Comuns aos Sistemas}
O sistemas de irrigação que utilizam fluidos presurizados compartilham de componentes comuns, dentre os quais são subdivididos quanto ao seu material, resistência, vazão e pressão de trabalho. Assim sua escolha apresenta dependencia explicita quanto ao tipo de sistema a ser implementado no cultivo.



\subsubsection{Tubulações}
Constituinte imprescindivel nos sistemas de irrigação, as tubulações apresentam caracteristicas peculiares quanto a seu material e resistência.

As empresas lideres do mercado disponibilizam catálogos técnicos contendo em seu bojo informações essenciais para o projetista, auxiliando na tomada de decisão de forma a promover excelência econômica, porém principalmente hidráulica, promovendo maior eficiência quanto aos parâmetros ideais para distribuição uniforme da água na área.
\subsubsection{Bombas}

\subsubsection{Filtros e Válculas}


\subsection{Microaspersão}

\subsection{Aspersão Convencional}

\subsection{Gotejamento}


\section{Utilização dos Sistemas}

\section{Ferramentas Similares}



\chapter{Informática}

\section{Ambiente Web}
\subsection{Métodos de Estruturação}

\subsection{Front-End}
\subsubsection{HTML}

\subsubsection{CSS}
A linguagem CSS é definida por ... . Ela foi utilizada de forma a promover responsividade a ferramenta, permitindo acesso as suas funcionalidades de maneira prática em qualquer dispositivo com capacidade de processamento reduzido, exigindo apenas acesso a internet. Assim, a estrutura necessária para o perfeito funcionamento é o Browser, o qual realizará om requerimento, transmissão e decodificação dos arquivos constituintes do sistema.
\subsubsection{JavaScript}

\subsection{Back-End}
\subsubsection{PHP}

\subsubsection{PHPMyAdmin}

\section{IDE}
\subsection{NotPad++}

\subsection{USBWServer}
\subsubsection{Apache}

\subsubsection{MySQL}

\section{Gráfica}
\subsection{CorelDraw}


\chapter{Portabilidade}

\section{Responsividade}


\section{Informações em Nuvem}



\bibliographystyle{apalike}
\bibliography{abntex2-options}

\printindex
\end{document}